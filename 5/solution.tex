\documentclass{article}
\usepackage{mathtools}
\usepackage{harpoon}
\usepackage{amsfonts}

\makeatletter
\newcommand{\tpmod}[1]{{\@displayfalse\pmod{#1}}}
\makeatother

\begin{document}
\title{Introduction to Modern Cryptography - EX. 5}
\author{Roi Koren 305428369\\ Noam Koren 204175004}
\date{\today}
\pagenumbering{gobble}
\maketitle
\newpage
\pagenumbering{roman}

\section{Commitments}
\begin{paragraph}
    a Even if computationally-unbounded, Bob can learn nothing about $s$.
    Since $g_2$ is a generator in $\mathbb{G}$, for each $s$ exists $r$ s.t. $g_2^r = hg_1^{-s}$.
    From this it follows that given $C$, Bob can only see a random value, hence learning nothing.
\end{paragraph}

\begin{paragraph}
    b Alice can't change $s$ without Bob noticing. Assume by contradiction that she can, i.e. Alice
    can calculate $s', r'$ s.t. $comm_r(s) = comm_{r'}(s') \Rightarrow g_1^sg_2^r = g_1^{s'}g_2^{r'} \Rightarrow
    g_1 = g_2^\frac{r' - r}{s - s'}$. This is in contradiction to the assumption that DLOG is hard in $\mathbb{G}$.
    Of course, if Alice is computationally-unbounded she can calculate $t$ s.t. $g_1 = g_2^t$ and switch
    the value of $(s, r)$ to $(s', r')$ while keeping that $t = \frac{r' - r}{s - s'}$.
\end{paragraph}

\begin{paragraph}
    c Checking if $p, q$ are primes of length larger then 512 bits is trivial, and that $p = 2q + 1$.
    To check if $g_1, g_2 \in \mathbb{Z}^*_p$ are of order $q$, since $\varphi(p) = 2q$, it is suffice to check that
    $g_{1,2}^q = 1\tpmod p$ and $g_{1,2}^2 \neq 1 \tpmod p$.
\end{paragraph}

\begin{paragraph}
    d Alice can indeed cheat, given the ability to cheat the scheme's parameters.
    Alice can choose $g_1, g_2$ s.t. $g_1 = g_2^t$. If after publishing a commitment $comm_r(s) = C = g_1^sg_2^r \tpmod p$,
    she would like to reveal $(s', r')$ instead of $(s, r)$, she can calculate $r' = r + t(s - s')$. And indeed, Bob will 
    calculate $C' = g_1^{s'}g_2^{r'} = g_1^{s'}g_2^{r + t(s-s')} = g_1^{s'}g_2^{t(s-s')}g_2^r = g_1^sg_2^r = C$.
\end{paragraph}

\begin{paragraph}
    e As we've seen in section a, Bob learns nothing of $s$ from $C$. This is true for every set of parameters. Thereby, Bob
    can not cheat.
\end{paragraph}

\begin{paragraph}
    f As a result of the previous two sections, letting Bob choose the parameters will keep the protocol secure.
\end{paragraph}

\section{ZK}
\begin{paragraph}
    a 
\end{paragraph}

\section{ZK with QR}
\begin{paragraph}
    a 
\end{paragraph}

\section{Coin tossing}
\begin{paragraph}
    a 
\end{paragraph}

\section{Coin tossing with QR}
\begin{paragraph}
    a 
\end{paragraph}

\end{document}
