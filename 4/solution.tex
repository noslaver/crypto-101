\documentclass{article}
\usepackage{mathtools}
\usepackage{harpoon}
\usepackage{amsfonts}

\makeatletter
\newcommand{\tpmod}[1]{{\@displayfalse\pmod{#1}}}
\makeatother

\begin{document}
\title{Introduction to Modern Cryptography - EX. 4}
\author{Roi Koren 305428369\\ Noam Koren 204175004}
\date{\today}
\pagenumbering{gobble}
\maketitle
\newpage
\pagenumbering{roman}

\section{Rabin's signatures}
\begin{paragraph}
    b The signer calculates $f(x) = x^{\frac{(p-1)(q-1)}{2}}$ and returns $x \in QR_n \\\iff f(x) = 1$.
\end{paragraph}

\begin{paragraph}
    c First we note that -
    \begin{equation*}
        p \equiv 3 \tpmod 8 \Rightarrow p \equiv 3 \tpmod 4
    \end{equation*}
    and that -
    \begin{equation*}
        q \equiv 7 \tpmod 8 \Rightarrow q \equiv 3 \tpmod 4  
    \end{equation*}
    As we've seen, we can then calculate - $x_p = \pm l^{\frac{p+1}{4}}$, 
    $x_q = \pm l^{\frac{q+1}{4}}$. The square roots are $(x_p, x_q), (-x_p, x_q), 
    (x_p, -x_q), (-x_p, -x_q)$.
\end{paragraph}

\begin{paragraph}
    d The verifier receives $(m, \sqrt{l})$. He then calculates $l$ and accepts if
    it is equal to any of $m, -m, 2m, -2m$.
\end{paragraph}

\begin{paragraph}
    e We'll split for 2 cases. {\bf1)} $l = \pm m$. Clearly, the same signature holds for the 4 messages $m, -m, \frac{1}{2}m, -\frac{1}{2}m$. 
    {\bf2)} $l = \pm 2m$. Clearly, the same signature holds for the 4 messages $m, -m, 2m, -2m$.
\end{paragraph}

\begin{paragraph}
    f If $m \in (\frac{n}{8}, \frac{n}{4})$ the other possible messages which may have
    the same signature belong to the following intervals - 
    $2m \in (\frac{n}{4}, \frac{n}{2})$, $-2m \in (\frac{n}{2}, \frac{3n}{4})$,
    $-m \in (\frac{3n}{4}, \frac{7n}{8})$, $\frac{m}{2} \in (\frac{n}{16}, \frac{n}{8})$,
    $\frac{m}{2} \in (\frac{7n}{8}, \frac{15n}{16})$. Notice all of the intervals are pairwise disjoint.
    The verification process can now include a check whether the received message $m$,
    is in the appropriate interval.
\end{paragraph}

\begin{paragraph}
    g The scheme is not secured under such a notion. An attacker can request two legal pairs - 
    $(m_1, \sqrt{l_1}), (m_2, \sqrt{l_2})$. He can then build a new message $m = l_1 \cdot l_2$,
    whose signature is $Sign(m) = \sqrt{l_1 \cdot \l_2} = \sqrt{l_1}\cdot \sqrt{l_2}$. Notice that $l_1, l_2 \in QR_n \Rightarrow l_1 \cdot l_2 \in QR_n$.
\end{paragraph}

\begin{paragraph}
    h HELLO
\end{paragraph}

\section{Schnorr signature}
\quad a 

\section{RSA signatures via CRT}
\quad a 

\section{Commitments}
\quad a

\section{ZK and Commitments}
\quad a

\section{Fun with secret sharing}
\quad a

\end{document}
