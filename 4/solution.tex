\documentclass{article}
\usepackage{mathtools}
\usepackage{harpoon}
\usepackage{amsfonts}

\makeatletter
\newcommand{\tpmod}[1]{{\@displayfalse\pmod{#1}}}
\makeatother

\begin{document}
\title{Introduction to Modern Cryptography - EX. 4}
\author{Roi Koren 305428369\\ Noam Koren 204175004}
\date{\today}
\pagenumbering{gobble}
\maketitle
\newpage
\pagenumbering{roman}

\section{Rabin's signatures}
\begin{paragraph}
    b The signer calculates $f(x) = x^{\frac{(p-1)(q-1)}{2}}$ and returns $x \in QR_n \\\iff f(x) = 1$.
\end{paragraph}

\begin{paragraph}
    c First we note that -
    \begin{equation*}
        p \equiv 3 \tpmod 8 \Rightarrow p \equiv 3 \tpmod 4
    \end{equation*}
    and that -
    \begin{equation*}
        q \equiv 7 \tpmod 8 \Rightarrow q \equiv 3 \tpmod 4  
    \end{equation*}
    As we've seen, we can then calculate - $x_p = \pm l^{\frac{p+1}{4}}$, 
    $x_q = \pm l^{\frac{q+1}{4}}$. The square roots are $(x_p, x_q), (-x_p, x_q), 
    (x_p, -x_q), (-x_p, -x_q)$.
\end{paragraph}

\begin{paragraph}
    d The verifier receives $(m, \sqrt{l})$. He then calculates $l$ and accepts if
    it is equal to any of $m, -m, 2m, -2m$.
\end{paragraph}

\begin{paragraph}
    e We'll split for 2 cases. {\bf1)} $l = \pm m$. Clearly, the same signature holds for the 4 messages $m, -m, \frac{1}{2}m, -\frac{1}{2}m$. 
    {\bf2)} $l = \pm 2m$. Clearly, the same signature holds for the 4 messages $m, -m, 2m, -2m$.
\end{paragraph}

\begin{paragraph}
    f If $m \in (\frac{n}{8}, \frac{n}{4})$ the other possible messages which may have
    the same signature belong to the following intervals - 
    $2m \in (\frac{n}{4}, \frac{n}{2})$, $-2m \in (\frac{n}{2}, \frac{3n}{4})$,
    $-m \in (\frac{3n}{4}, \frac{7n}{8})$, $\frac{m}{2} \in (\frac{n}{16}, \frac{n}{8})$,
    $\frac{m}{2} \in (\frac{7n}{8}, \frac{15n}{16})$. Notice all of the intervals are pairwise disjoint.
    The verification process can now include a check whether the received message $m$,
    is in the appropriate interval.
\end{paragraph}

\begin{paragraph}
    g The scheme is not secured under such a notion. An attacker can request two legal pairs - 
    $(m_1, \sqrt{l_1}), (m_2, \sqrt{l_2})$. He can then build a new message $m = l_1 \cdot l_2$,
    whose signature is $Sign(m) = \sqrt{l_1 \cdot l_2} = \sqrt{l_1}\cdot \sqrt{l_2}$.
    Notice that $l_1, l_2 \in QR_n \Rightarrow l_1 \cdot l_2 \in QR_n$ meaning the given signature
    is indeed a valid signature.
\end{paragraph}

\begin{paragraph}
    h In order to build a secure signature scheme we assume we have a random oracle $\mathcal{H}$.
    The signature scheme would be $\sigma(m) = \sqrt{\mathcal{H}(m)}$.
    Assume towards a contradiction that there exists an adversary $\mathcal{A}$ of complexity $(t, \varepsilon)$,
    that can query $\mathcal{H}$ and the signer and is able to forge a valid signature. We'll build an adversary
    $\mathcal{B}$ that is able of factoring. When $\mathcal{A}$ requests queries from $\mathcal{H}$, if that value was
    previously requested, respond as before, else, choose a random value $x \in (\frac{n}{8}, \frac{n}{4})$,
    save the result and return $x^2$. When $\mathcal{A}$ requests a signature for $m$, if that value was
    previously requested, respond as before, else, choose a random value $x \in (\frac{n}{8}, \frac{n}{4})$,
    save the result and return $x$. \\
    After $t$ queries, $\mathcal{A}$ will return a legal message-signature pair, $(m, \sigma)$ with probability $\varepsilon$.
    Since when $\mathcal{A}$ requested $\mathcal{H}$, we returned $x^2 \Rightarrow \sigma = \sqrt{(x^2)}$. This
    combined with $x$ which we saved gives us two roots of $x^2$, which we can use to factorize
    $\mathcal{H}(m)$ w.p $\frac{1}{2}$. Hence, forging a signature is as hard as factoring.
\end{paragraph}

\section{Schnorr signature}
\begin{paragraph}
    a The modification is as follows: to sign a message \(m\), send \((b, c)\), when we have \(a = g^r, b = H(a, m), c = r + xb\), with \(r\) being chosen at random from \(\mathbb{Z}_q\).
    
    To verify, one will first compute \(a = g^c \cdot (h^b)^{-1}\), then verify that \(b = H(a, m)\).\\
    
    We have
    \begin{align*}
        a' = g^c \cdot(h^b)^{-1} = g^{r + xb} \cdot(h^b)^{-1} = g^r \cdot g^{xb} \cdot(h^b)^{-1} = g^r \cdot h^b \cdot(h^b)^{-1} = g^r
    \end{align*}
    
    Thus, the \(a\) we computed is equal to the signer's \(a\) that was not sent, and correctness follows. We need only exponentiate twice in \(\mathbb{Z}_q\) and find an inverse once, as well as call the global hash function \(H\) once, all of which can be done in \(\log q\) time, meaning the verification process can be computed efficiently.
\end{paragraph}

\begin{paragraph}
    b To do this, we'll modify the public key to equal \(h^{-1}\). Using the previous scheme, one will now calculate \(a = g^c \cdot h'^b\), with the rest of the scheme staying the same.
    \begin{align*}
        a' = g^c \cdot h'^b = g^{r + xb} \cdot(h^{-1})^b = g^r \cdot g^{xb} \cdot(h^b)^{-1} = g^r \cdot h^b \cdot(h^b)^{-1} = g^r
    \end{align*}
    
    Once again, correctness quickly follows. We have performed only 2 exponentiations, and did not require finding an inverse, as required.
\end{paragraph}

\begin{paragraph}
    c Given two signatures, \((a = g^r, b, c = r + xb)\) and \((a, b', c' = r + xb')\), one can simply calculate, since with high probability \(b \neq b'\)
    \begin{align*}
        \frac{c - c'}{b - b'} = \frac{r + xb - r - xb'}{b - b'} = \frac{x(b - b')}{b - b'} = x
    \end{align*}
    
    And find \(x\) thus.
\end{paragraph}

\begin{paragraph}
    d Given two signatures, of two different known messages, with consecutive \(r\)-s, \((a = g^r, b, c = r + xb)\) and \((a' = g^{r + 1} = g \cdot g^r, b' c' = (r + 1) + xb')\), since again \(b \neq b'\), we get
    \begin{align*}
        \frac{c - c' + 1}{b - b'} = \frac{r + xb - r - 1 - xb' + 1}{b - b'} = \frac{x(b - b')}{b - b'} = x
    \end{align*}
    
    Again, quickly finding \(x\)
\end{paragraph}
\newpage

\begin{paragraph}
    e Given three signatures, \((a_1, b_1, c_1 = r_1 + xb_1 = r_0 + C + xb_1)\), \((a_2, b_2, c_2 = r_2 + xb_2 = r_0 + 2C + xb_2)\) and \((a_3, b_3, c_3 = r_3 + xb_3 = r_0 + 3C + xb_3)\), we will calculate
    \begin{align*}
        \frac{c_3 + c_1 - 2c_2}{b_3 + b_1 - 2b_2} &= \frac{r_0 + 3C + xb_3 + r_0 + C + xb_1 - 2(r_0 + 2C + xb_2)}{b_3 + b_1 - 2b_2}\\
        &= \frac{x(b_3 + b_1 - 2b_2)}{b_3 + b_1 - 2b_2} = x
    \end{align*}
    
    Again, finding \(x\) rather quickly.
\end{paragraph}

\section{RSA signatures via CRT}
\begin{paragraph}
    a Given an error calculating $S_q$, gives $S_q' = S_q + \varepsilon$, one gets the following
    corrupted signature - $S' = q\cdot inv(q, p)\cdot S_p + p\cdot inv(p,q)\cdot S_q' =
    S + p\cdot inv(p,q)\cdot \varepsilon$, where $S$ is the correct signature. Given both $S, S'$,
    one can calculate their difference - $S' - S = p\cdot \varepsilon\cdot inv(p,q) \Rightarrow p|S'-S$.
    Using $gcd(S' - S, N)$ we can find $p$.
\end{paragraph}

\begin{paragraph}
    b TODO
\end{paragraph}

\section{Commitments}
\begin{paragraph}
    a Given \((y, c)\), for Bob to be able to identify \(b\) means he can distinguish between \(f(r)\) and \(f(r) \odot y\). \(r\) is uniformly distributed over \(\mathbb{G}\), and so \(f(r)\) is as well. We also know that \(f(r) \odot y = f(r) \odot f(x) = f(r \odot x)\) is also uniformly distributed over \(\mathbb{G}\), since \(r \odot x\) is. Bob cannot efficiently distinguish between two uniformly distributed random variables, and so cannot identify \(b\).
\end{paragraph}

\begin{paragraph}
    b Continuing on the previous section, if Bob could find for the given \(c\) an \(x'\) satisfying \(f(x') = c\), he could still not identify \(b\). This is because he will be left with one of two numbers, \(r\) and \(r \odot x\), both of which are uniformly distributed over \(\mathbb{G}\). Thus, again, Bob cannot discern \(b\).
\end{paragraph}

\begin{paragraph}
    c Assume Alice managed to cheat, and found a commitment \(c\), and two reveals \((r_0, 0)\) and \((r_1, 1)\) that both open the commitment. It means that:
    \begin{align*}
        c &= f(r_0)\\
        c &= y \odot f(r_1) = f(x) \odot f(r_1) = f(x \odot r_1)
        f(r_0) &= f(x \odot r_1)
    \end{align*}
    
    Since \(f\) is a permutation, a one-to-one function, this means that the above becomes \(r_0 = x \odot r_1 \rightarrow x = r_0 \odot r_1\). Alice cannot find \(x\) efficiently, since that would mean to find the inverse of \(f\), and so cannot efficiently find a commitment \(c\) and two reveals \((r_0, 0)\) and \((r_1, 1)\) that both open it, and therefor cannot cheat.
\end{paragraph}

\begin{paragraph}
    d Bob should not agree to this. From the previous section, if Alice chooses \(y\), i.e. picks \(x\) at random and calculates \(y = f(x)\), she knows how to generate a commitment \(c\), and two reveals \((r_0, 0)\) and \((r_1, 1)\) that both open the commitment. She only needs to pick a random \(r_0\), and set \(r_1 = x \odot r_0\), and send the commitment \(c = f(r_0)\).
    
    If Alice wants to reveal 0, she will send \((r_0, 0)\), and if she wants to reveal 1, she'll send \((r_1, 1)\).
    \begin{align*}
        f(r_1) \odot y = f(r_1) \odot f(x) = f(r_1 \odot x) = f(x \odot r_0 \odot x) = f(r_0) = c
    \end{align*}
    
    Meaning Alice can cheat, and Bob should not agree to her suggestion.
\end{paragraph}

\begin{paragraph}
    e Alice can agree, if she wants to. If Bob chooses \(y\), and therefor knows \(x\), he still cannot distinguish between \(f(r_0)\) and \(y \odot f(r_1) = f(x) \odot f(r_1) = f(x \odot r_1)\), at least not without finding the inverse of \(f\), which is assumed he cannot do efficiently. Thus Alice can agree to Bob's suggestion.
\end{paragraph}

\begin{paragraph}
    f If Bob gets to pick the parameters, it means he knows the factorization of \(n\). This means that he can pick \(e = 2\), and choose a \(y \in QNR^+_{pq}\).
    
    When Bob receives a commitment \(c\), he can efficiently check if \(c \in QR_{pq}\). If \(b = 0\), it must be that \(c = f(r) = r^2 \in QR_{pq}\). Otherwise, as we've seen in recitation, \(c = y \cdot f(r) = y \cdot r^2 \not\in QR_{pq}\). Thus Bob knows that \(b = 0 \iff c \in QR_{pq}\), and he can open the commitment without Alice's reveal.
\end{paragraph}

\section{ZK and Commitments}
\begin{paragraph}
    a Bob will accept with $\Pr = 1$. Since $N \in L$, $y$ has exactly four square roots, which
    Alice can find efficiently, given that she knows $p,q$ s.t. $N = p\cdot q$, one
    of which must be the one sampled by Bob, meaning Bob will receive a list which follows
    the protocol.
\end{paragraph}

\begin{paragraph}
    b As we've seen in class $\mathbb{Z}^*_{pqr} \cong \mathbb{Z}^*_p \times \mathbb{Z}^*_q \times \mathbb{Z}^*_r$.
    This means that any number $y$ from the group will have 8 square roots.
    Bob will accept only if $x$ is one of the roots that Alice chooses. Since Alice can't
    know which is the root chosen by Bob, she can only guess 4 random roots.
    The probability that $x$ is among those roots is $\frac{4}{8} = \frac{1}{2}$, hence
    Bob accepts with $\Pr = \frac{1}{2}$.
\end{paragraph}

\begin{paragraph}
    c The protocol isn't sound, since as we've seen, Bob accepts a wrong answer w.p. $\frac{1}{2}$.
    To improve soundness, one can repeat the protocol $n$ times, which will decrease the probability
    of accepting a false statement to $2^{-n}$.
\end{paragraph}

\begin{paragraph}
    d TODO
\end{paragraph}

\section{Fun with secret sharing}
\begin{paragraph}
    a Suppose w.l.g that Alice is the honest colonel. Alice sees - 
    \begin{equation*}
    f =
        \begin{cases}
            f(Alice) + f(Bob) + r, & \mbox{w.p } \frac{1}{2} \\
            f(Alice) + f(Carol) + r, & \mbox{w.p } \frac{1}{2}
        \end{cases}
    \end{equation*}
    Where $r$ is an unknown random element. Since both outcomes appear random to Alice,
    differentiating between the two, is equivalent to her differentiating between two random elements,
    which she cannot.
\end{paragraph}

\begin{paragraph}
    b An outside investigator can try all possible trios of colonels, of which there are
    $\binom{5}{3} = 10$. Each colonel is part of 6 trios, meaning the result will be 4 correct
    answers for the trios where the cheater is not present, and the cheating colonel is the one
    not present in any of those trios.
\end{paragraph}

\begin{paragraph}
    c A modification of this scheme can include a commitment by each of the colonels, where by
    each colonel commits to his share when the secret is shared. When a launch is requested,
    each colonel will reveal to make sure his secret hasn't been modified.
\end{paragraph}

\end{document}
