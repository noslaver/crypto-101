\documentclass{article}
\usepackage{mathtools}
\usepackage{harpoon}
\usepackage{pagecolor}

\pagecolor{black}
\color{white}

\begin{document}
\title{Introduction to Modern Cryptography - EX. 2}
\author{Roi Koren 305428369\\ Noam Koren 204175004}
\date{\today}
\pagenumbering{gobble}
\maketitle
\newpage
\pagenumbering{roman}

\section{Feistel Networks \& DES}
\begin{paragraph}
	a For a round-function $f = 0$ of a Feistel network, we get that for the i-th round, with
	input $(L_i, R_i)$ the output of $(R_i, L_i\oplus f(R_i, K_i)) = (R_i, L_i\oplus 0) = 
	(R_i, L_i)$. So for a r-round network, the output will be $(L_0, R_0)$ for an even r, and
	$(R_0, L_0)$ for an odd r.
\end{paragraph}

\begin{paragraph}
	b For a round-function $f(x) = x$ we'll get the following outputs for first three rounds - 
	$y_1 = (R_0, L_0\oplus R_0)$, $y_2 = (L_0\oplus R_0, L_0)$, $y_3 = (L_0, R_0)$. What we find
	is a 3 round cycle. So for a r-round network we get -
	\begin{equation*}
	y_r = 
		\begin{cases}
			(L_0, R_0), & \mbox{if } r \;\bmod\; 3 = 0 \\
			(R_0, L_0\oplus R_0), & \mbox{if } r \;\bmod\; 3 = 1 \\
			(L_0\oplus R_0, L_0), & \mbox{if } r \;\bmod\; 3 = 2 
		\end{cases}	
	\end{equation*}
\end{paragraph}

\begin{paragraph}
	c First we'll compute Feistel$_{f_1,f_2}(L_0, R_0)$ -
	\begin{align*}
		y_1 &= (R_0, L_0\oplus f_1(R_0)) \\
		y_2 &= {Feistel}_{f_1,f_2}(L_0, R_0) = (L_0\oplus f_1(R_0), R_0\oplus f_2(L_0\oplus f_1(R_0)))
	\end{align*}
	
	Now, for Feistel$_{f_2,f_1}(R_2, L_2)$. After one round we get - 
	\begin{align*}
		y'_1 &= (L_0\oplus f_1(R_0), R_0\oplus f_2(L_0\oplus f_1(R_0))\oplus f_2(L_0\oplus f_1(R_0))) \\
	  &= (L_0\oplus f_1(R_0), R_0)
	\end{align*}
	After the second round we get - 
	\begin{align*}
		y'_2 &= (R_0, L_0\oplus f_1(R_0) \oplus f_1(R_0)) = (R_0, L_0)
	\end{align*}
\end{paragraph}

\begin{paragraph}
	d DES is a 16-round Feistel network, where the round functions are determined
	by a subset of the 56-bit key. Therefore, that $DES_k^{-1}$ is equal to
	$DES_k$ with the order of the keys reversed, is to say that if Feistel$_{f_1,..,f_{16}}(L_0, R_0) = (L_{16},R_{16})$ than 
	Feistel$_{f_{16},..f_1}(R_{16},L_{16}) = (R_0, L_0)$. This is a generalized result of
	section $(c)$. We'll prove by induction. The base case is covered be section
	$(c)$. Assume correctness for Feistel$_{f_1,..f_n}$ - 
	\begin{align*}
		&Feistel_{f_1,..,f_n}(L_0,R_0) = (L_n,R_n) \\
		&\Rightarrow Feistel_{f_n,..,f_1}(R_n,L_n) = (R_0, L_0)
	\end{align*}
	The induction step - passing the result of a n-round network through another round
	will result in
	\begin{align*}
		Feistel_{f_1,f_{n+1}}(L_0,R_0) = (R_n, L_n\oplus f_{n+1}(R_n))
	\end{align*}
	Passing The reversed binary string to the reversed Feistel network we get - 
	\begin{align*}
		Feistel_{f_{n+1},..f_1}(L_n\oplus f_{n+1}(R_n),R_n) = (R_n,L_n)
	\end{align*}
	This proves that for any n-round Feistel network, and specifically for a 16-round DES
	decryption is the same as encryption with the round-functions reversed.
\end{paragraph}

\begin{paragraph}
	e My attack is a meet-in-the-middle attack. With a given plaintext and ciphertext -
	$(M_1, E(M_1))$, we go through the 28-bit key space, running 8 rounds of
	DES encryption. This will give $2^{28}$ possible half-encrypted data. Next we go again
	through the 28-bit key space, this time half-decrypting the given ciphertext 8-rounds
	of DES. Giving us $2^{28}$ possible half-decrypted data. Finally we search through the 
	resulting arrays, searching for a key which gave the same result.
\end{paragraph}

\section{Enhancing DES}
\begin{paragraph}
	{$DES^V:$} Assume we have a moderate $\geq$ 2 number of messages, $M_i$, and their encryptions, $C_i = DES^V_{k, k_1}(M_i) = DES_k(M_i) \oplus k_1$. Do the following for each possible value of $k$, denoted $k'$:
	
	Encrypt $M_1$ using $DES_{k'}$, and get $k'_1 = DES_k(M_1) \oplus k_1 \oplus DES_{k'}(M_1)$. Next calculate $C'_2 = DES_k(M_1) \oplus k_1 \oplus k'_1$, and decrypt it using $DES_{k'}$. If we got $M_2$ at the end, it means the odds of 
	$k = k', k_1 = k'_1$. We can run $DES^V_{k', k'_1}$ on a few more messages, if we want to be certain that we found the right pair of keys. Since we are going solely over the possibilities for the first key $k$, it will take in the order of $2^{56}$ $DES$ encryptions and decryptions to find the correct pair of keys.
\end{paragraph}

\begin{paragraph}
    {$DES^{W}:$} Again assume a moderate $\geq$ 2 number of messages $M_i$ and their encryptions $C_{i} = DES^{W}_{k, k_1}(M_i) = DES_{k}(M_i \oplus k_1)$. For each possible value of $k$, denoted again as $k'$, do the following:
    
    Decrypt $C_1$ using $DES_{k'}$, and calculate $k'_1 = DES^{-1}_{k'}(DES_k(M_1 \oplus k_1)) \oplus M_1$. Then encrypt $M_2$ using $DES^W_{k', k'_1}$, getting $C'_2 = DES_{k'}(M_2 \oplus k'_1)$. If we get $C'_2 = C_2$, it means again that $k' = k, k'_1 = k_1$. Once more we can run $DES^W_{k', k'_1}$ to assure ourselves that we got the right combination of keys. As was the case with $DES^V$, we need in the order of $2^{56}$ $DES$ encryptions/decryptions to break the cipher.
\end{paragraph}

\section{Tweaking AES}
\begin{paragraph}
	a The SubBytes stage of the AES algorithm is its source of non-linearity.
	Hence, removing this stage makes the 'tweaked' AES linear.
	We can build a set of 128 linear equations for each bit of the output. And after 128 inputs
	solve that set. After that we can distinguish between a random function and the modified-AES
	by checking if the cipher-text adheres to that set of equations.
\end{paragraph}

\begin{paragraph}
	b The MixColumns stage of the AES algorithm is a source of diffusion.
	An attack on this modification may give two inputs that have the same first-row and
	check whether the outputs have an identical row.
\end{paragraph}

\begin{paragraph}
	c The ShiftRows stage of the AES algorithm is a source of diffusion.
	Similar to the previous, replacing rows with columns, an attack on this modification
	 may give two inputs that have the same first-column and check whether the outputs
	 have an identical column.
\end{paragraph}

\begin{paragraph}
	d Combining all operations of the same type significantly weakens the AES scheme.
	The updated AES is equivalent to ${E(M) = M'\oplus K}$, where $M'$ is a computable function
	on $M$, comprised of the first three stages - 10 SubBytes, 10 ShiftRows and 9 MixColumns, which
	are independent of the secret key. And $K$ is the XOR of all the subkeys
	from the 11 AddRoundKey operations. Since we can compute $M'$ ourselves, this scheme is 
	the same as using a secret key to XOR your messages, which we've seen to be unsafe.
	An example of an attack would be to pass a message $M$ and then xor-ing the output with
	$M'$ to compute $K$. For a following message we do the same and check if we get the 
	same key, allowing us the distinguish between the modified AES and a random function.

\end{paragraph}

\section{MACs}
\begin{paragraph}
	a To break the $OFB-MAC$, one would only have to query once, on $m$ and receive $OFB-MAC_k(m) = E_k(...(E_k(S_0)...) \oplus m$, and output a new pair $(m' \neq m, OFB-MAC_k(m) \oplus m \oplus m')$, where $|m| = |m'|$.
	\begin{align*}
	    OFB-MAC_k(m) \oplus m \oplus m' &= E_k(...(E_k(S_0)...) \oplus m \oplus m \oplus m'\\
	    &= E_k(...(E_k(S_0)...) \oplus m'\\
	    &= OFB-MAC_k(m')
	\end{align*}
	
	We have successfully outputted a new pair $(m', OFB-MAC_k(m'))$, meaning we have broken the $OFB-MAC$.
\end{paragraph}

\begin{paragraph}
    b To break this $CBC-MAC$, one would have to go the extra mile, and make 3 queries. Query the $MAC$ on three messages, namely $m_1, m_2, m_3 = m_1 || 1 || m_4$, with $m_1 \neq m_2, \quad |m_1| = |m_2| = |m_4| = 1$ block, and getting back
    \begin{align*}
        CBC-MAC_k(m_1, |m_1|) &= t_1\\
        CBC-MAC_k(m_2, |m_2|) &= t_2\\
        CBC-MAC_k(m_3, |m_3|) &= t_3
    \end{align*}
    
    Output a new pair $(m', t')$, with $m' = m_2 || 1 || m_4 \oplus t_1 \oplus t_2$, and $t' = t_2$. Note here that $||$ means concatenation, and the number 1 is represented by a block's length of bits, according to $E$.
    \begin{align*}
        E_k(E_k(m_2) \oplus 1) &= t_2\\
        E_k(E_k(m_1) \oplus 1) &= t_1\\
        E_k(E_k(E_k(m_2) \oplus 1) \oplus (m_4 \oplus t_1 \oplus t_2)) &= E_k(t_2 \oplus m_4 \oplus t_1 \oplus t_2)\\
        &= E_k(m_4 \oplus t_1)\\
        E_k(E_k(E_k(m_1) \oplus 1) \oplus m_4) &= E_k(t_1 \oplus m_4)
    \end{align*}
    
    Since $|m'| = |m_3|$, $CBC-MAC_k(m', |m'|) = CBC-MAC_k(m_3, |m_3|)$. Thus we have outputted a valid new pair, breaking this $CBC-MAC$.
\end{paragraph}

\begin{paragraph}
    c For this $CBC-MAC$, we're back to needing a single query to break it. Query the $MAC$ on $m$, a single block message, to get back $(S_0, t)$. Now output a new pair $(m', (S'_0, t))$, with $m' = m \oplus S_0 \oplus S'_0$ and $S'_0$ being a new initialization vector, different than $S_0$. Assuming the chosen initialization vector is $S'_0$, we get
    \begin{align*}
        CBC-MAC_k^{S'_0}(m') &= CBC-MAC_k^{S'_0}(m \oplus S_0 \oplus S'_0)\\
        &= E_k(m \oplus S_0 \oplus S'_0 \oplus S'_0)\\
        &= E_k(m \oplus S_0)\\
        &= CBC-MAC_k^{S_0}(m')
    \end{align*}
    
    From this, and since the initialization vector is passed with the result of the $CBC-MAC$ as part of the tag, we get that the new pair is valid, and this modification to $CBC-MAC$ is broken as well.
\end{paragraph}

\begin{paragraph}
    d As we've seen in class, and will show in $Q6$, to find two different messages with the same tag, one needs in the order of $2^{\frac{n}{2} + 1} = 2^{65}$ queries on different messages to find 2 messages with the same tag, with probability $\geq \frac{1}{2}$. Specifically, we'll query on random messages of a fixed length, that end with a 1. Suppose we have found two such messages, $M, M'$. We can assume that, since $t = t'$, $CBC-MAC_{k_1}(M) = CBC-MAC_{k_1}(M')$, because otherwise 2 different message get the same encryption under $E_{k_2}$, a contradiction to it being decrypt-able. We'll query the $ECBC-MAC$ once more, on $\Tilde{M} = M || 0^{128}$, to get
    \begin{align*}
        ECBC-MAC_{k_1, k_2}(\Tilde{M}) &= E_{k_2}(CBC-MAC_{k_1}(\Tilde{M}))\\
        &= E_{k_2}(E_{k_1}(CBC-MAC_{k_1}(M) \oplus 0^{128}))\\
        &= E_{k_2}(E_{k_1}(CBC-MAC_{k_1}(M))) = \Tilde{t}
    \end{align*}
    
    Finally, we'll output a new pair $(M^* = M' || 0^{128}, \Tilde{t})$. Notice we haven't queried the $ECBC-MAC$ on $M^*$, and that
    \begin{align*}
        ECBC-MAC_{k_1, k_2}(M^*) &= E_{k_2}(CBC-MAC_{k_1}(M^*))\\
        &= E_{k_2}(E_{k_1}(CBC-MAC_{k_1}(M') \oplus 0^{128}))\\
        &= E_{k_2}(E_{k_1}(CBC-MAC_{k_1}(M')))\\
        &= E_{k_2}(E_{k_1}(CBC-MAC_{k_1}(M))) = \Tilde{t}
    \end{align*}
    
    Hence, the new pair is valid. In conclusion, with complexity $2^{65}$ and probability $\geq \frac{1}{2}$, we have broken the $ECBC-MAC$.
\end{paragraph}

\section{Extending hash functions from fixed length to variable length messages}
\begin{paragraph}
	a Note that $H(M) = h(y_s, m_s)$. Meaning that finding $M_1,M_2$ such that $H(M_1) = H(M_2)$
	is equivalent to finding $h(y^1_s, m^1_s) = h(y^2_s, m^2_s)$.
	An adversary $\mathcal{A}'$ can use the same algorithm $\mathcal{A}$ uses for $s = 1$, and output
	$y_1m_1, y_2m_2$.
	Since $\mathcal{A}'$ does one iteration where $\mathcal{A}$ does $s$ iterations, its
	complexity is $\frac{t_A}{s}$.
\end{paragraph}

\section{CBC-MAC}
\begin{paragraph}
	a Let $K$ be a key of length $n$ and let $(m_1, m_2, m_3)$ be chosen uniformly over $\{0, 1\}^n$. It follows that the random variable $C_1 = E_K(m_1)$ is also distributed uniformly over the same $\{0, 1\}^n$, assuming $E$ is a truly random function. Therefore $C_2 = C_1 \oplus m_2$ and $C_3 = E_K(C_2)$ are also uniformly distributed. More importantly, since $m_3$ is distributed uniformly, so are $C_4 = m_3 \oplus C_3$ and $\sigma = E_K(C_3)$.
\end{paragraph}

\begin{paragraph}
    b From the first part of this question, $C = E_k(E_k(m_1) \oplus m_2)$ has a 1 in $2^{2n}$ probability for having any specific value. From the birthday "paradox" we know that the odds of two such $C$'s (from two different messages) getting the same value are $\frac{1}{2^{2n}}$. The odds then of there being such a collision for $d$ different messages is equal to 1 minus the odds of there being 0 collisions, which in itself is equal to:
    \begin{align*}
        \Pr[\forall i \neq j. C_{M_i} \neq C_{M_j}] &=^{(*)} \prod\limits_{1 \leq i < j \leq d}\Pr[C_{M_i} \neq C_{M_j}]\\
        &= \prod\limits_{i = 1}^d \left(1 - \frac{i}{2^{2n}}\right) \leq \prod\limits_{i = 1}^d e^{\frac{-i}{2^{2n}}}\\
        &= e^{\sum\limits_{i = 1}^d \frac{-i}{2^{2n}}} = e^{-\frac{d(d + 1)}{2^{2n+1}}}
    \end{align*}
    
    In $^{(*)}$ we used the fact that the different messages and their $C$'s are i.i.d. We want those odds to be $\leq \frac{1}{2}$, and so we get:
    \begin{align*}
        e^{-\frac{d(d + 1)}{2^{2n+1}}} &\leq \frac{1}{2} \quad |\ln(\cdot)\\
        -\frac{d(d + 1)}{2^{2n+1}} &\leq -\ln2\\
        \frac{d(d + 1)}{2^{2n+1}} &\geq \ln2\\
        d^2 + d &\geq \ln2 \cdot 2^{2n+1}
    \end{align*}
    
    For $d \geq \sqrt{2\ln2} \cdot 2^n$, the inequality holds. Thus that number of different messages will suffice to find a collision with probability $\geq \frac{1}{2}$.
\end{paragraph}

\begin{paragraph}
    c Going with the clue, our adversary $\mathcal{A}$ will ask around $c + 9$ queries for messages of the form $M = (m_1, m_2, 0^n)$. From the previous part of the question, with probability $\geq \frac{1}{2}$, we have found a pair of messages, $M = (m_1, m_2, 0^n)$ and $M' = (m'_1, m'_2, 0^n)$ such that
    \begin{align*}
        \sigma = E_k(0^n \oplus E_k(m_2 \oplus E_k(m_1))) = E_k(0^n \oplus E_k(m'_2 \oplus E_k(m'_1))) = \sigma'
    \end{align*}
    
    If no collisions were found, $\mathcal{A}$ will just give up, and return $(M^*, \sigma^*)$, for some random message and tag.\\
    
    The adversary will next ask for $\Tilde{\sigma}$, the tag of $\Tilde{M} = (m_1, m_2, 1^n)$, and output $\left(M^* = (m'_1, m'_2, 1^n), \Tilde{\sigma}\right)$. From the way it was constructed, $\mathcal{A}$ hadn't queried on $\hat{M}$, and we know that
    \begin{align*}
        \hat{\sigma} = E_k(E_k(E_k(m'_1) \oplus m'_2) \oplus 1^n) = E_k(E_k(E_k(m_1) \oplus m_2) \oplus 1^n) = \sigma^*
    \end{align*}
    
    This means $\mathcal{A}$ has outputted a pair $(M^*, \sigma^*)$, which with probability $\geq \frac{1}{2}$ is new and valid. QED.
\end{paragraph}

\end{document}
