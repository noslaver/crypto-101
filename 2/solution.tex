\documentclass{article}
\usepackage{mathtools}
\usepackage{harpoon}
\usepackage{pagecolor}

\pagecolor{black}
\color{white}

\begin{document}
\title{Introduction to Modern Cryptography - EX. 2}
\author{Roi Koren 305428369\\ Noam Koren 204175004}
\date{\today}
\pagenumbering{gobble}
\maketitle
\newpage
\pagenumbering{roman}

\section{Feistel Networks \& DES}
\begin{paragraph}
	a For a round-function $f = 0$ of a Feistel network, we get that for the i-th round, with
	input $(L_i, R_i)$ the output of $(R_i, L_i\oplus f(R_i, K_i)) = (R_i, L_i\oplus 0) = 
	(R_i, L_i)$. So for a r-round network, the output will be $(L_0, R_0)$ for an even r, and
	$(R_0, L_0)$ for an odd r.
\end{paragraph}

\begin{paragraph}
	b For a round-function $f(x) = x$ we'll get the following outputs for first three rounds - 
	$y_1 = (R_0, L_0\oplus R_0)$, $y_2 = (L_0\oplus R_0, L_0)$, $y_3 = (L_0, R_0)$. What we find
	is a 3 round cycle. So for a r-round network we get -
	\begin{equation*}
	y_r = 
		\begin{cases}
			(L_0, R_0), & \mbox{if } r \;\bmod\; 3 = 0 \\
			(R_0, L_0\oplus R_0), & \mbox{if } r \;\bmod\; 3 = 1 \\
			(L_0\oplus R_0, L_0), & \mbox{if } r \;\bmod\; 3 = 2 
		\end{cases}	
	\end{equation*}
\end{paragraph}

\begin{paragraph}
	c First we'll compute Feistel$_{f_1,f_2}(L_0, R_0)$ -
	\begin{align*}
		y_1 &= (R_0, L_0\oplus f_1(R_0)) \\
		y_2 &= {Feistel}_{f_1,f_2}(L_0, R_0) = (L_0\oplus f_1(R_0), R_0\oplus f_2(L_0\oplus f_1(R_0)))
	\end{align*}
	
	Now, for Feistel$_{f_2,f_1}(R_2, L_2)$. After one round we get - 
	\begin{align*}
		y'_1 &= (L_0\oplus f_1(R_0), R_0\oplus f_2(L_0\oplus f_1(R_0))\oplus f_2(L_0\oplus f_1(R_0))) \\
	  &= (L_0\oplus f_1(R_0), R_0)
	\end{align*}
	After the second round we get - 
	\begin{align*}
		y'_2 &= (R_0, L_0\oplus f_1(R_0) \oplus f_1(R_0)) = (R_0, L_0)
	\end{align*}
\end{paragraph}

\begin{paragraph}
	d DES is a 16-round Feistel network.
\end{paragraph}

\begin{paragraph}
	e H
\end{paragraph}

\section{Enhancing DES}
\begin{paragraph}
	a Placeholder
\end{paragraph}

\section{Tweaking AES}
\begin{paragraph}
	a The SubBytes stage of the AES algorithm is its source of non-linearity.
	Hence, removing this stage makes the 'tweaked' AES linear.
	We can build a set of 128 linear equations for each bit of the output. And after 128 inputs
	solve that set. After that we can distinguish between a random function and the modified-AES
	by checking if the cipher-text adheres to that set of equations.
\end{paragraph}

\begin{paragraph}
	b The MixColumns stage of the AES algorithm is a source of diffusion.
	An attack on this modification may give two inputs that have the same first-row and
	check whether the outputs have an identical row.
\end{paragraph}

\begin{paragraph}
	c The ShiftRows stage of the AES algorithm is a source of diffusion.
	Similar to the previous, replacing rows with columns, an attack on this modification
	 may give two inputs that have the same first-column and check whether the outputs
	 have an identical column.
\end{paragraph}

\begin{paragraph}
	d Combining all operations of the same type significantly weakens the AES scheme.
	The updated AES is equivalent to ${E(M) = M'\oplus K}$, where $M'$ is a computable function
	on $M$, comprised of the first three stages - 10 SubBytes, 10 ShiftRows and 9 MixColumns, which
	are independent of the secret key. And $K$ is the XOR of all the subkeys
	from the 11 AddRoundKey operations. Since we can compute $M'$ ourselves, this scheme is 
	the same as using a secret key to XOR your messages, which we've seen to be unsafe.
	An example of an attack would be to pass a message $M$ and then xor-ing the output with
	$M'$ to compute $K$. For a following message we do the same and check if we get the 
	same key, allowing us the distinguish between the modified AES and a random function.

\end{paragraph}

\section{MACs}
\begin{paragraph}
	a Placeholder
\end{paragraph}

\section{Extending hash functions from fixed length to variable length messages}
\begin{paragraph}
	a Note that $H(M) = h(y_s, m_s)$. Meaning that finding $M_1,M_2$ such that $H(M_1) = H(M_2)$
	is equivalent to finding $h(y^1_s, m^1_s) = h(y^2_s, m^2_s)$.
	An adversary $\mathcal{A}'$ can use the same algorithm $\mathcal{A}$ uses for $s = 1$, and output
	$y_1m_1, y_2m_2$.
	Since $\mathcal{A}'$ does one iteration where $\mathcal{A}$ does $s$ iterations, its
	complexity is $\frac{t_A}{s}$.
\end{paragraph}

\section{CBC-MAC}
\begin{paragraph}
	a Placeholder
\end{paragraph}

\end{document}
